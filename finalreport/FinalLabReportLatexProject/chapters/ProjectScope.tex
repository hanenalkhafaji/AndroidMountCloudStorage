%!TEX root = ../Report.tex
\chapter{Project Scope} \cite{scopetemplate}
\label{projscope}

\section{Introduction}
This chapter outlines the scope of the project that is being explored, researched, and implemented for CS 6970. It includes the description, scope, and deliverables of the project as well as details on what criteria is expected to be met in order for this project to be deemed accepted. Another important purpose for this document is so that all stakeholders involved in this project have a common understanding of the scope, expectations, and goals of this project. August 5, 2019 is the expected completion date and submission of all deliverables for this project.

\section{Project Purpose and Justification}
The Cloud Storage Mounting on Android OS project has been approved for researching, planning, designing, building, and implementation. The project involves mounting a cloud storage directory at the Operating System level so that the directory can be accessed from other applications running on the mobile device. The purpose is to make that cloud storage appear to be native to the device rather than hosted elsewhere. The successful implementation of this project will meet the requirements for the course CS 6970 as well as produce software that will greatly reduce the friction of cloud storage access for mobile device users.

\section{Scope Description}
The scope of the Cloud Storage Mounting on Android OS project will involve research, planning, designing, building, and implementing a solution for mounting a cloud storage at the Operating System level of an Android mobile device. Open source resources will be used as tools for implementation as well as starting points.

\section{High Level Requirements}
The only requirement specified for this particular project is to properly perform mounting of a cloud storage directory so that it can be accessed from anywhere on the mobile device.

\section{Boundaries}
The scope of the Cloud Storage Mounting on Android OS project includes all work involving research, planning, designing, building, and implementing a solution for mounting a cloud storage at the Operating System level of an Android mobile device. Tasks that will also be involved include gathering requirements, writing up requested documents and documentation as well as the technical report, deploying the solution, and testing the solution on an Android mobile device. The scope of this project will not include an implementation for Windows or Apple mobile devices or implementations for other cloud storage providers outside of Google Drive, unless time permits otherwise. 

\section{Strategy}
The Cloud Storage Mounting on Android OS project will be implemented by one developer using a machine running Linux and has Android Studio installed. The developer will research open source options to use as a template to begin the implementation. The implementation will be tested locally on the Linux machine to debug and make sure functionality is operating as expected. Then, either by using the mobile device emulator or the physical Android device (ASUS Nexus 7 tablet), the implementation will be tested on a mobile device for completeness. 

\section{Deliverables}
The Cloud Storage Mounting on Android OS project will yield an Android APK file that can be downloaded and installed on an Android mobile device. A technical report detailing the project will be delivered as well. All deliverables will be upload onto GitHub \cite{projectgithub}.

\section{Acceptance Criteria}
\begin{itemize}
  \item Meet scheduled deadlines.
  \item Develop a report that documents successes and failures during the project.
  \item Store all project artifacts in a GitHub account. 
  \item Complete at least 3 labs accompanied by respective lab reports.
  \item Demonstrate progressive learning of project components.
\end{itemize}

\section{Constraints}
The project must be completed between May 13, 2019 and August 5, 2019. That is about 12 weeks. Development will be performed on a personal Linux (Ubuntu 18.10) machine. Testing will be performed on the same machine and an ASUS Nexus 7 tablet that is running Android version 6.0.1 and has been rooted.

\section{Assumptions}
The application will be installed on only Android devices. The user on the device has root access. The directory being mounted is on the Google Cloud provider, not any other provider.