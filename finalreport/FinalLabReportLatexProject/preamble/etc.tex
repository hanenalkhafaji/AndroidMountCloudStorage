%!TEX root = ../Report.tex 

% Content specific packages.

\usepackage{blindtext}
\usepackage{algorithm}
\usepackage{algpseudocode}
%\usepackage{pgfplots}                 % Plot tools
%\usetikzlibrary{
%    arrows,
%    matrix,
%    positioning,
%    shapes,
%    topaths,
%}
%\pgfplotsset{compat=1.7}

% Listings
\lstset{
    basicstyle=\footnotesize\ttfamily,% the size of the fonts that are used for the code
    breakatwhitespace=false,          % sets if automatic breaks should only happen at whitespace
    breaklines=true,                  % sets automatic line breaking
    captionpos=b,                     % sets the caption-position to bottom
    commentstyle=\color{s14a},        % comment style
    deletekeywords={},                % if you want to delete keywords from the given language
    escapeinside={\%*}{*)},           % if you want to add LaTeX within your code
    frame=single,                     % adds a frame around the code
    keywordstyle=\bfseries\ttfamily\color{s09}, % keyword style
    language=Python,                  % the language of the code
    morekeywords={*,...},             % if you want to add more keywords to the set
    numbers=left,                     % where to put the line-numbers; possible values are (none, left, right)
    numbersep=5pt,                    % how far the line-numbers are from the code
    numberstyle=\sffamily\tiny\color{dtugray}, % the style that is used for the line-numbers
    rulecolor=\color{dtugray},        % if not set, the frame-color may be changed on line-breaks within not-black text (e.g. comments (green here))
    showspaces=false,                 % show spaces everywhere adding particular underscores; it overrides 'showstringspaces'
    showstringspaces=false,           % underline spaces within strings only
    showtabs=false,                   % show tabs within strings adding particular underscores
    stepnumber=1,                     % the step between two line-numbers. If it's 1, each line will be numbered
    stringstyle=\color{s07},          % string literal style
    tabsize=2,                        % sets default tab size to 2 spaces
    title=\lstname,                   % show the file name of files included with \lstinputlisting; also try caption instead of title
}

\usepackage{pdflscape}
\usepackage{acronym}
%\usepackage{natbib}
\usepackage{amsmath}
\usepackage{enumitem}
\usepackage{lipsum}  