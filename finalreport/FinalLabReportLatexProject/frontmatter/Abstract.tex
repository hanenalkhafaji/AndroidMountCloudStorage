%!TEX root = ../Report.tex
\chapter{Abstract}
Currently, most cloud storage applications available for Android devices simply give the user an interface in which to sync his/her files and folders. There is not an application that will mount the cloud storage directory so that it appears to be part of the Operating System's file system. This would allow the user to access their cloud based directories and files from any application. For the file the user needs to access, he/she would avoid having to download it to the local file system in order to allow another application on the device access to that file. Once that application has completed it's tasks against that file, the user would then need to remember to sync that file back to their cloud storage in order for that file to remain up-to-date. This is inconvenient, a hassle, and vulnerable to human error. \\ \\
The scenario described above is the motivation for the project upon which this course and final paper are based. How can I alleviate those extra syncing steps and allow the user to access their cloud storage content directly? The solution is to mount a cloud storage directory to the internal storage of an Android device, at the Operating System level. I journeyed toward this goal with the understanding that open source projects were available with solutions close to what I was trying to accomplish. This shed hope on the implementation aspect of the project. I could take what was already created and build upon it. This, unfortunately, proved to be more difficult than I initially had envisioned. \\ \\
Many of the solutions were created for the purpose of mounting on Linux alone. Even still, this seemed to be a great starting point since Android is built on the Linux Operating System. In saying that, porting a solution that works on Linux over to Android should not be difficult. However, I ran into issues due to unfamiliar territory. Specifically, cross compiling in order to obtain an Android compatible executable. I ran into several obstacles along the way that did not allow me to achieve the ultimate goal of mounting the cloud storage directory on an Android device at the Operating System level. This attempt was based solely on my experience with OCamlFUSE which was written in OCaml and was designed to mount a Google Drive directory on Linux. \\ \\ 
I did several other activities outside of the OCamlFUSE project in an effort to explore other options. Once I had decided that OCamlFUSE was no longer an option, I needed to turn my attention to something else. It was beneficial that I was trying other options in conjunction with OCamlFUSE. As a result of this side work, I had discovered RClone and was able to install it onto my Android device. I was able to successfully mount my Google Drive directory to the Android file system. While working with RClone, I also discovered that it supports many other cloud storage providers. So, I went on to explore the other options available within RClone. \\ \\
It continues to be unsettling that the OCamlFUSE solution could not easily be ported over to the Android device. Future work should include continuing to dig into cross compilers for OCaml and attempting to compile OCamlFUSE for use on an Android device. There are also several other solutions that could be explored. I feel that I only grazed the surface of cloud storage mounting solutions and cross compilation solutions available on GitHub. Given more time, I would have experimented with more solutions to find out which could crack the code for OCamlFUSE.